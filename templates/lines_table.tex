%\documentclass[12pt,a4paper,twoside]{article}
\documentclass{standalone}
\usepackage[utf8]{inputenc}
\usepackage[T1]{fontenc}
\usepackage[english]{babel}
\usepackage{lmodern}
\usepackage{hyperref}
\usepackage{xcolor}
\usepackage{amsmath}
\usepackage{amssymb}
\usepackage{mathrsfs}
\usepackage{array}
\usepackage{tabularx}
\usepackage{multirow}
\usepackage{longtable}
\usepackage[nomessages]{fp}

\begin{document}
	\FPeval{\pointsTypeNumber}{clip({{pointsTypeNumber}})}%
	\FPeval{\clineEnd}{clip({{pointsTypeNumber}}+4)}%
	\noindent
	\begin{tabular}{|c|c|c|c|c|c|}
		\hline
		\multirow{2}{*}{Id} & \multirow{2}{*}{Projective} & \multicolumn{2}{c|}{Core} & \multicolumn{\pointsTypeNumber}{c|}{Composition} & \multirow{2}{*}{cardinal}\\
		\cline{3-\clineEnd}
		 & & Points & Lines &  H{{count(i)}} &  \\
		\hline
		\hline

		{{count(index)}} & {{line/isProjective}} & {{line/core/points}} & {{line/core/lines}} &  {{val}} &  {{line/cardinal}}\\
		\hline

	\end{tabular}
\end{document}
